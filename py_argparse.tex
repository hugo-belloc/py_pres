\documentclass{beamer}

\usepackage[T1]{fontenc}
\usepackage[utf8]{inputenc}
\usepackage[frenchb]{babel}
\usepackage{minted} % For python code inclusion
\usepackage{tcolorbox}
\usepackage{default}



\definecolor{ttybg}{RGB}{48,10,36}
\definecolor{ttyfg}{RGB}{255,241,200}

% Insert slide number
\addtobeamertemplate{navigation symbols}{}{%
    \usebeamerfont{footline}%
    \usebeamercolor[fg]{footline}%
    \hspace{1em}%
    \insertframenumber/\inserttotalframenumber
}

    
% Configure beamer
\usetheme{Warsaw}
\hypersetup{pdfpagemode=FullScreen}
\title[Le module argparse]{Créer des interfaces ligne de commande avec arguments en Python}
\author{Hugo Belloc}
\date{\today}

\begin{document}

\begin{frame}
 \titlepage
\end{frame}

\begin{frame}
 \tableofcontents
\end{frame}

% \section{Introduction}
% \subsection{Les interfaces ligne de commande}
% 
% \begin{frame}
%  \frametitle{Interfaces lignes de commande et interfaces graphiques}
%  
%  \begin{definition}
%   A utiliser pour donner une définition.
%  \end{definition}
%  
%  
% \end{frame}
% 
% 
% \subsection{Pourquoi utiliser les interfaces lignes de commande avec argument ?}
% 
% \subsection{Le module argparse}

\section{Les bases - analyse d'un exemple minimal}
\subsection{Le code}

\begin{frame}
 \frametitle{Code minimal}
   
\inputminted[fontsize=\fontsize{8}{9}]{python}{py_src/py_argparse/argparse_minimal.py}

\end{frame}

\subsection{Explications}

\begin{frame}
 \frametitle{Explication rapide}
   
\inputminted[linenos=true, fontsize=\fontsize{3}{4}, frame=single]{python}{py_src/py_argparse/argparse_minimal.py}

\begin{description}
 \item[Ligne 6] Définition du parseur
 \item[Lignes 7-8] Définition des arguments à lire
 \item[Ligne 9] Analyse de la ligne de commande
 \item[Lignes 10-11] Utilisation des arguments extraits
\end{description}

\end{frame}

\subsection{Utilisation}

\begin{frame}[fragile]
 \frametitle{Affichage de l'aide}
 
  \begin{tcolorbox}[left=2pt, width=\textwidth,colback={ttybg},title={Console}, colupper=ttyfg]
  \begin{scriptsize}

  \begin{verbatim}
  $ python argparse_minimal.py --help
  usage: argparse_minimal.py [-h] [-o OPTIONAL_ARG] positional_arg

  Exemple de parseur ligne de commande

  positional arguments:
    positional_arg        Arg positionnel

  optional arguments:
    -h, --help            show this help message and exit
    -o OPTIONAL_ARG, --optional-arg OPTIONAL_ARG
			  Arg optionnel

  > Process finished with exit code 0 

  \end{verbatim}
  \end{scriptsize}
  
 \end{tcolorbox}

\end{frame}

\section{Notions avancées}

\section{Pour aller plus loin}


\end{document}