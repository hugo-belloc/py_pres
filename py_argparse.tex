\documentclass{beamer}

\usepackage[T1]{fontenc}
\usepackage[utf8]{inputenc}
\usepackage[frenchb]{babel}
\usepackage{minted}
\usepackage{default}

% Minimise python code
%\newminted{python}{fontsize=\footnotesize}

\usetheme{Warsaw}
\hypersetup{pdfpagemode=FullScreen}

\title[Le module argparse]{Créer des interfaces ligne de commande avec arguments en Python}
\author{Hugo Belloc}
\date{\today}

% Insert frame number
\addtobeamertemplate{navigation symbols}{}{%
    \usebeamerfont{footline}%
    \usebeamercolor[fg]{footline}%
    \hspace{1em}%
    \insertframenumber/\inserttotalframenumber
}


\begin{document}

\newmintedfile{python}{
  linenos=true,
  fontsize=\fontsize{5}{6},
}


\begin{frame}
 \titlepage
\end{frame}

\begin{frame}
 \tableofcontents
\end{frame}

% \section{Introduction}
% \subsection{Les interfaces ligne de commande}
% 
% \begin{frame}
%  \frametitle{Interfaces lignes de commande et interfaces graphiques}
%  
%  \begin{definition}
%   A utiliser pour donner une définition.
%  \end{definition}
%  
%  
% \end{frame}
% 
% 
% \subsection{Pourquoi utiliser les interfaces lignes de commande avec argument ?}
% 
% \subsection{Comment faire en Python?}

\section{Exemple}
\begin{frame}
 \frametitle{Code minimal}
 
 Mon code:
  
\inputminted[linenos=true, fontsize=\fontsize{7}{8}]{python}{py_src/py_argparse/argparse_minimal.py}

\end{frame}





\end{document}